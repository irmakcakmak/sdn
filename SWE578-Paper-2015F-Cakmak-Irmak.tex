\documentclass[12pt,journal,compsoc]{IEEEtran}

\usepackage{multirow}
\usepackage{graphicx}
% *** CITATION PACKAGES ***
%
\ifCLASSOPTIONcompsoc
  % The IEEE Computer Society needs nocompress option
  % requires cite.sty v4.0 or later (November 2003)
  \usepackage[nocompress]{cite}
\else
  % normal IEEE
  \usepackage{cite}
\fi
% cite.sty was written by Donald Arseneau
% V1.6 and later of IEEEtran pre-defines the format of the cite.sty package
% \cite{} output to follow that of the IEEE. Loading the cite package will
% result in citation numbers being automatically sorted and properly
% "compressed/ranged". e.g., [1], [9], [2], [7], [5], [6] without using
% cite.sty will become [1], [2], [5]--[7], [9] using cite.sty. cite.sty's
% \cite will automatically add leading space, if needed. Use cite.sty's
% noadjust option (cite.sty V3.8 and later) if you want to turn this off
% such as if a citation ever needs to be enclosed in parenthesis.
% cite.sty is already installed on most LaTeX systems. Be sure and use
% version 5.0 (2009-03-20) and later if using hyperref.sty.
% The latest version can be obtained at:
% http://www.ctan.org/pkg/cite
% The documentation is contained in the cite.sty file itself.
%
% Note that some packages require special options to format as the Computer
% Society requires. In particular, Computer Society  papers do not use
% compressed citation ranges as is done in typical IEEE papers
% (e.g., [1]-[4]). Instead, they list every citation separately in order
% (e.g., [1], [2], [3], [4]). To get the latter we need to load the cite
% package with the nocompress option which is supported by cite.sty v4.0
% and later.





\ifCLASSINFOpdf
\else
\fi

\newcommand\MYhyperrefoptions{bookmarks=true,bookmarksnumbered=true,
pdfpagemode={UseOutlines},plainpages=false,pdfpagelabels=true,
colorlinks=true,linkcolor={black},citecolor={black},urlcolor={black},
pdftitle={Bare Demo of IEEEtran.cls for Computer Society Journals},%<!CHANGE!
pdfsubject={Typesetting},%<!CHANGE!
pdfauthor={Michael D. Shell},%<!CHANGE!
pdfkeywords={Computer Society, IEEEtran, journal, LaTeX, paper,
             template}}%<^!CHANGE!

\hyphenation{op-tical net-works semi-conduc-tor}


\begin{document}

\title{A Survey on Software Defined Networking }

\author{Irmak~Cakmak,~\IEEEmembership{irmak@fastmail.com}}

\markboth{Bogazici University SWE578 Course, December~2015}%
{Shell \MakeLowercase{\textit{et al.}}: Bare Advanced Demo of IEEEtran.cls for IEEE Computer Society Journals}

\IEEEtitleabstractindextext{%
\begin{abstract}
Software defined networking is a design model, set of principles, and an abstraction rather than a new technology. By defining right abstractions, this approach makes network design, implementation, operation, debugging, and troubleshooting easier. Also by providing layering, it makes innovation in network technologies easier and more cost effective. This short paper tries to describe how and whys of these claims shortly, from the perspective of the author.
\end{abstract}

\begin{IEEEkeywords}
Software Defined Networking
\end{IEEEkeywords}}

\maketitle

\IEEEdisplaynontitleabstractindextext

\IEEEpeerreviewmaketitle

\ifCLASSOPTIONcompsoc
\IEEEraisesectionheading{\section{Introduction}\label{sec:introduction}}
\else
\section{Introduction}
\label{sec:introduction}
\fi

\IEEEPARstart{C}{urrent} IP based network technologies have three abstractions: data plane, control plane, and management plane. Data plane's responsibility is described by simply forwarding of packets through switches or routers. In case of a switch forwarding, network device simply checks the header of the packet and then forwards the packet to its destination via its MAC address. Routers do the same job through IP address based forwarding. This process is simple as no state computation is done. On the contrary, control plane computes the next forwarding state, which means whether the packet will be forwarded, dropped or rerouted to another destination than its intended target. Example mechanisms for control plane are firewalls, distributed routing algorithms, ACLs, VLANs, MPLS, etc. Management plane carries administration traffic. Commands like telnet, ping, traceroute, etc. are trafficked through this plane. Generally management plane is considered within the control plane. The argument of SDN is data plane has a very well abstraction and theoretical base; yet, control plane is a bag of protocols lacking any abstraction, layering or theoretical background which hinders innovation in control plane and in general, networking. TCP/IP technology, for instance, has very good abstractions as seven layers. These layers are developed further based on the theory behind it, without affecting other layers, and without needing to rethink every aspect of the protocol. On the other side, if a researcher is designing a new algorithm for routing, she needs to start from scratch. SDN argues, providing right abstractions for control plane will make it open to innovation and development, similar to data plane. 

\section{SDN}
In current networking architectures, both control plane and data plane are generally 
bundled together in network switches. When a new device is added to the network 
or a policy is changed, operator needs to configure each device individually. 
Also, there exist a lot of middle-boxes, such as firewalls or packet inspectors, 
which also enforce control policies to the network. These also require individual 
administration and configuration. This situation makes management of large 
networks very hard, error prone, difficult to troubleshoot. As stated before, 
SDN argues, control plane needs better abstractions. What are these 
abstractions? First control plane needs to discover the network topology 
to achieve its goals. Then, it needs to find out how it can enforce its 
goal (firewall isolation, trafficking, routing) on that topology. 
Finally, it needs to describe switches and routers how to forward. To 
be able to achieve these goals, SDN provides two abstractions: 
network discovery and network switch configuration.

Network discovery is done by network operating system. It runs on commodity 
hardware servers within the network and communicates with switches through 
a forwarding model. It is aware of the network topology through a 
virtualized network abstraction (network virtualization). NOS 
configures network switches for forwarding through southbound APIs 
(One of the API providers is OpenFlow and it will be described 
in the next section). Operator programs the network using a control 
program. Simply, operator writes her function and control 
program executes the function against the network topology, using 
network operating system’s APIs. The abstraction provided here is 
that control program does not concern itself with the whereabouts of 
the network topology. It is provided by the operating system. 
On the other side, OS does not concern itself with the goal of 
the operator. It provides the network topology and conveys the control 
program’s forwarding information down to the switches. 
This clean separation of concerns allows innovation in different 
layers keeping them intact. 



\section{Main Components of an SDN~Architecture}

\subsubsection{Infrastructure}
Network infrastructure is typically composed of switches, routers, and various middle-boxes.
Generally all these devices are more intelligent than they need to be. They decide what
packet to forward and what not to, meaning that assuming the role of control plane elements.
In an SDN network architecture, as mentioned before, simple data forwarding operation and 
deciding what to forward are seperated. In other words, intelligence is removed from data 
plane and given to a centralized control system, generally called, network operating system 
(hereafter, NOS). As a consequence, what is left for infrastructure elements are simple 
packet forwarding job. Yet, control programs need to be able to program those devices, 
and this is typically done by using a common protocol (e.g. OpenFlow), providing 
programmability of different devices, along with interoperability of such devices 
in a heterogeneous networking environment.

Programming APIs, mediating communication between forwarding and control elements
are called Southbound interfaces.
Simply, it sits between the NOS and hardware in the architecture. Topology changes,
 flow statistics, and alien packets seen first time by switches are provided to 
NOS by southbound interfaces.

OpenFlow is such a protocol  for programming
 switches and routers exploiting common denominator of device interfaces exposed.
It provides set of methods for programming flow-table of devices.
An OpenFlow switch has three pieces: flow table, secure channel, 
and OpenFlow protocol. Flow table keeps the flow entries, which holds the actions 
for telling a device how to process the flow. Secure channel is the communication 
path between the device and the control program, through which commands and packets are 
transferred. Lastly, OpenFlow protocol supplies the means of communication between 
the control program and the device.

Definition of flow is generally dependent on the context. In one network environment it 
can be defined as all packets from a specific IP or MAC address, yet in another 
context it can be defined as all packets originating from the same port of a router. 
In more general sense, it can be defined as a group of packets meeting a certain criteria.

Flow table entries have three fields: definition header, action, and statistics. 
Fields of a definition header can be seen in table one. Actions defines the way to 
process matching packets. Basic examples for actions are "forward matching packets to 
specific ports", "make controller program receive this flow's packets", and 
"drop packets belonging to a flow". Statistics fields keeps basic accounting of flows. 
These stats include the number of packets matching each flow and the timestamp of 
the last matched packet to a flow.

\begin{table}[]
\centering
\begin{tabular}{|c|c|c|c|c|c|c|c|c|c|}
\hline
\multirow{2}{*}{\begin{tabular}[c]{@{}c@{}}In\\ Port\end{tabular}} & \multirow{2}{*}{\begin{tabular}[c]{@{}c@{}}VLAN\\ ID\end{tabular}} & \multicolumn{3}{c|}{Ethernet} & \multicolumn{3}{c|}{IP} & \multicolumn{2}{c|}{TCP} \\ \cline{3-10} 
 &  & SA & DA & Type & SA & DA & Proto & Src & Dst \\ \hline
\end{tabular}
\caption{Flow Header}
\label{my-label}
\end{table}


\subsubsection{Network Operating System}
Network operating systems exist for quite some time, but they all have been specific
 to certain vendors and closed source. Also they have been more like a collection 
of instructions (Juniper JunOS, Cisco IOS).

On the other hand, NOSs in SDN nomenclature, have wider responsibilities and capabilities.
 They sit between the southbound interface and network control applications.
 Their goals are similar to traditional OS, but in a networking 
level. NOSs provide two basic novelties: first, applications on top of NOS can 
be programmed using a centralized programming model, viewing the whole network as 
a single machine; second, they can be programmed using higher level
abstractions (host names, user names) instead of low-level primitives (MAC address).
 A network programmer does not need to know how to interface with any forwarding 
device.

NOS views network in switch level along with middle-boxes, users, offered services.
 It only lacks current state of network traffic. NOS gathers this view using 
the aforementioned flows provided through southbound interfaces. In detail, when a
 flow in a switch does not match any entry in the table, it is forwarded 
to the control program (NOS). Alternatively, flow-table might have been programmed 
to forward packets belonging to particular protocols, such as DNS, to NOS. Using 
these flows NOS gathers the network view and decides to forward the traffic or not
 and calculates the route.

 NOX, Floodlight, Maestro, Beacon, ProgrammableFlow, Ryu NOS, Rosemary are some 
examples of such control progragrams, addressing different needs and having 
different purposes. All are considered as network operating systems.

Programming APIs provided by a NOS are generally called "Northbound interface"s.
 Their goal are similar to what POSIX tries to achieve for UNIX, providing a 
standard way of programming to OS programmers. These APIs are RESTful in some 
instances. On some others, they are provided as programming languages or file 
systems.

Also network operating systems which have distributed nature, expose east and 
westbound interfaces providing functions for data communication among 
cntroller instances, applying data consistency algorithms, and controller node 
monitoring.

\subsubsection{Applications}
 Network applications are the actual entites managing the network. Proposed 
layering up until now is solely aimed at providing right abstractions for 
applications. Forwarding behaviour of the network is programmed in application 
level against the virtual network topology provided by network operating system,
 network virtualization, and higher level programming languages. Defined behaviour 
is installed to switches by NOS through southbound API.

Network applications are developed and deployed for executing 
certain network functions such as, 
enforcing security policies, routing, network virtualization, load balancing,
 power saving, providing reliability, measurement and monitoring,
 traffic engineering and optimization, etc.

One of the most promising novelites provided by this capability in 
networking is SDN app stores. HP hosts an app store similar to Apple or
 Google's stores for mobile devices. Store provides apps for OpenFlow
 based HP devices.
\section{Conclusion}
There is a tremendous research effort ongoing in the field of SDN. Research
 and development is focused on improving switch design for performance, 
OpenFlow compatibility, and flow table capacity; operating system for better
 application portability, modularity, availability, scalability, resiliency, 
and security. Concurrent research ongoing in all these fields are enabled by
 the opportunity provided by SDN's novel approach to networking. SDN's 
offerings are seperation of data and control planes, central network view, and 
dynamic programmability of the network, which form the core of this novelty.

\ifCLASSOPTIONcompsoc
  % The Computer Society usually uses the plural form
  \section*{Acknowledgments}
\else
  % regular IEEE prefers the singular form
  \section*{Acknowledgment}
\fi


The author would like to thank Prof.Dr. Fatih Alagoz for supporting this learning opportunity.


\ifCLASSOPTIONcaptionsoff
  \newpage
\fi


\begin{thebibliography}{1}

\bibitem{IEEEhowto:kopka}
H.~Kopka and P.~W. Daly, \emph{A Guide to {\LaTeX}}, 3rd~ed.\hskip 1em plus
  0.5em minus 0.4em\relax Harlow, England: Addison-Wesley, 1999.

\end{thebibliography}

\end{document}


